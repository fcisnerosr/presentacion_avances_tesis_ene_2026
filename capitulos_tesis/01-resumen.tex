\chapter*{Resumen}
\addcontentsline{toc}{chapter}{Resumen}
\label{chap:resumen}

La integridad estructural de las plataformas fijas tipo Jacket en la Sonda de Campeche es un desafío crítico debido a su antigüedad extendida y a las limitaciones de las inspecciones visuales submarinas tradicionales. Esta investigación desarrolla y valida una metodología de detección de daño no destructiva fundamentada en la \textbf{Fusión de Datos y Optimización Combinatoria}. A diferencia de los métodos tradicionales de Actualización de Modelos (\textit{Model Updating}), esta técnica emplea un Algoritmo Genético (AG) para calibrar un vector de pesos óptimos que integra múltiples índices de daño vibratorios en un indicador unificado, maximizando así la capacidad de localización de fallas sin alterar los parámetros físicos del modelo numérico.

El presente documento reporta el avance de la investigación doctoral en curso, detallando la formulación del marco computacional y su validación frente a corosión y abolladuras. Específicamente, se analiza la \textbf{corrosión generalizada}, con énfasis en la discrepancia entre las normas de diseño (NORSOK M-001) y las tasas reales en la zona de salpicaduras; y las \textbf{abolladuras mecánicas} en elementos tubulares. Los resultados obtenidos demuestran la capacidad del algoritmo para localizar y cuantificar estos daños, identificando a las diagonales principales como "fusibles de alerta" con alta sensibilidad vibratoria frente a la robustez de las piernas principales. Como parte de las siguientes etapas del proyecto doctoral, se integrará el análisis de grietas por fatiga en uniones soldadas y deformaciones globales excesivas, cuyos resultados completarán el alcance de esta tesis para consolidar una herramienta integral de Monitoreo de Salud Estructural (SHM).
