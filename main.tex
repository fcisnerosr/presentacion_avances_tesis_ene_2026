\documentclass{beamer}

\usetheme{Madrid}
\useoutertheme{miniframes}
\usecolortheme{whale}
\setbeamerfont{section in head/foot}{size=\fontsize{6}{7}\selectfont}

\usepackage[utf8]{inputenc}
\usepackage[spanish]{babel}
\decimalpoint
\usepackage{graphicx}
\PassOptionsToPackage{table}{xcolor}
\usepackage{subcaption}
\usepackage{booktabs}
\usepackage{colortbl}
\usepackage{array} % Para formateo avanzado de tablas
\usepackage{amsmath}

% Path configuration
\graphicspath{{./figs/}{./figs/resultados/abolladura/}{./figs/resultados/corrosion/}}

% Metadata
\title[Detección de Daño en Plataformas]{Desarrollo de una metodología para la detección de daño en plataformas marinas fijas por medio de análisis de vibraciones}
\author[M. I. Francisco Cisneros]{M. I. Francisco Cisneros \\ \small{Doctorante del 7mo Semestre}}
\institute[]{Posgrado en Ingeniería - UNAM}
\date{Viernes 23 de enero de 2026}

\begin{document}

% Estructura de secciones para barra de navegación
\section[Intro]{Introducción}

% Slide 1: Carátula Institucional Personalizada
\begin{frame}[plain]
    \centering
    % 1. Logo Institucional
    \includegraphics[height=1.3cm]{000_logo.png}
    \vspace{0.2cm}

    % 2. Encabezado Institucional
    \vspace{0.1cm}
    {\footnotesize Coordinación Académica del Posgrado} \\
    {\footnotesize Dirección de Desarrollo de Talento}

    \vspace{0.6cm}

    % 3. Contexto del Evento
    {\bfseries \textit{EVALUACIÓN DE DESEMPEÑO}} \\
    \vspace{0.1cm}
    Trabajo de Investigación de Doctorado

    \vspace{0.4cm}

    % 4. Título de la Tesis
    \begin{beamercolorbox}[sep=8pt,center,shadow=true,rounded=true]{title}
        \usebeamerfont{title}Desarrollo de una metodología para la detección de daño en plataformas marinas fijas por medio de análisis de vibraciones\par
    \end{beamercolorbox}

    \vspace{0.4cm}

    % 5. Autor
    \textbf{M. I. Francisco Cisneros}

    \vfill

    % 6. Pie de Página
    {\footnotesize
    Directores: Dr. Iván Félix González y Dr. Rolando Salgado Estrada \\
    \vspace{0.1cm}
    Periodo: Invierno 2025
    }
    \vspace{0.2cm}
\end{frame}

% Slide 2: Introducción y Alcance
\subsection{Alcance y Problemática}
\begin{frame}{Introducción y Alcance}
    \begin{itemize}
        \item \textbf{Contexto:} Avances finales previos a la defensa de tesis (Evaluación de 7º Semestre).
        \vspace{0.5cm}
        \item \textbf{Problemática:}
        \begin{itemize}
            \item Crisis de mantenimiento en infraestructura envejecida.
            \item "Data Scarcity": Escasez de datos reales de daño que inviabiliza el uso puro de algoritmos de Deep Learning.
        \end{itemize}
        \vspace{0.5cm}
        \item \textbf{Solución Propuesta:}
        \begin{itemize}
            \item Hibridación de Algoritmos Genéticos (AG) con Modelos de Elemento Finito (FEM).
        \end{itemize}
    \end{itemize}
\end{frame}

% Slide 3: Metodología Propuesta (1/3)
\section[Metodología]{Metodología}
\subsection{Flujo General}
\begin{frame}{Metodología Propuesta (1/3): Flujo General}
    \centering
    \includegraphics[width=\textwidth, height=0.62\textheight, keepaspectratio]{005_metogologia_propuesta_para_identificar_daños_en_plataformas_reales.jpg}
    \vspace{0.2cm}
    \captionof{figure}{Flujo de identificación en plataformas reales.}
\end{frame}

% Slide 4: Metodología Propuesta (2/3)
\subsection{Algoritmo Genético}
\begin{frame}{Metodología Propuesta (2/3): Esquema Computacional}
    \centering
    \includegraphics[width=\textwidth, height=0.62\textheight, keepaspectratio]{006_metodología_del_AG_para_localizar_daños_a_nivel_computacional.jpg}
    \vspace{0.2cm}
    \captionof{figure}{Esquema computacional del AG.}
\end{frame}

% Slide 5: Metodología Propuesta (3/3)
\begin{frame}{Metodología Propuesta (3/3): El Algoritmo Genético}
    \centering
    \includegraphics[width=\textwidth, height=0.62\textheight, keepaspectratio]{015_diagrama_como_funciona_el_AG_para_la_sumas_funcion_P_y_como_se_relaciona_con_operadores_geneticos_del_AG.jpg}
    \captionof{figure}{Mecánica del AG: función objetivo y operadores genéticos.}
\end{frame}


% Caso de Estudio (Modelo Jacket)
\section[Caso]{Caso de Estudio}
\begin{frame}{Caso de Estudio: Plataforma Tipo Jacket}
    \begin{itemize}
        \item Plataforma marina fija discretizada mediante elementos viga.
    \end{itemize}
    \vspace{0.3cm}
    \begin{columns}
        \column{0.5\textwidth}
        \begin{figure}
            \centering
            \includegraphics[width=\linewidth, height=0.7\textheight, keepaspectratio]{007_modelo_vista_fronta_de_plataforma_tipo_jacket_de_caso_de_estudio.png}
            \caption{Vista Frontal}
        \end{figure}
        
        \column{0.5\textwidth}
        \begin{figure}
            \centering
            \includegraphics[width=\linewidth, height=0.7\textheight, keepaspectratio]{007_modeo_vista_3d_de_plataforma_tipo_jacket_de_caso_de_estudio.png}
            \caption{Vista 3D}
        \end{figure}
    \end{columns}
\end{frame}

% --- NUEVA POSICIÓN: EFECTOS INERCIALES (Antes de Justificación) ---
\section[Efectos]{Efectos Inerciales y Ambientales}
\begin{frame}{Efectos Inerciales y Ambientales (1/2)}
    \textbf{Consideraciones:} Inclusión de masa añadida hidrodinámica y crecimiento marino (biofouling).
    \begin{figure}
        \centering
        \includegraphics[width=0.85\textwidth, height=0.52\textheight, keepaspectratio]{001_modelo_plataforma_con_elemento_tubular_masa_adherida_y_masa_atrapada.png}
        \caption{Detalle de Masa Hidrodinámica (Adherida y Atrapada).}
    \end{figure}
\end{frame}

\begin{frame}{Efectos Inerciales y Ambientales (2/2)}
    \begin{figure}
        \centering
        \includegraphics[width=0.85\textwidth, height=0.52\textheight, keepaspectratio]{001_modelo_plataforma_con_elemento_tubular_masa_de_crecimiento_marino.png}
        \caption{Modelado del Crecimiento Marino (Biofouling) y coeficientes rugosos.}
    \end{figure}
\end{frame}

% --- NUEVA POSICIÓN: JUSTIFICACIÓN DE DAÑOS ---
\section[Justificación]{Justificación de Daños}

% 1. CORROSIÓN (Primero)
\subsection{Corrosión}
\begin{frame}{Justificación de Daños: Fenomenología Zonal (1/2)}
    \small
    \textbf{¿Por qué modelar Corrosión Uniforme?}
    Aunque la corrosión puede ser local, para efectos de rigidez global ($EI$) en estructuras Jacket, la normativa (API/ISO) valida el modelo de \textbf{reducción uniforme de espesor} ($t_{loss}$) como el mecanismo que gobierna la pérdida de capacidad de carga.

    \vspace{0.3cm}
    \textbf{Distribución Vertical de la Severidad:}
    La pérdida de espesor \textbf{no es constante} en toda la altura. Se concentra drásticamente en la zona de humectación y secado (Splash Zone), creando un cuello de botella estructural.

    \vfill 
    \hrule \vspace{0.1cm}
    \tiny \color{gray}
    \textbf{Fuentes Normativas:} \\
    \textbullet\ \textbf{ISO 19902:2007:} \textit{Petroleum and natural gas industries -- Fixed steel offshore structures.} (Sec. 16.3 Structural Integrity). \\
    \textbullet\ \textbf{API RP 2A-WSD:} \textit{Recommended Practice for Planning, Designing and Constructing Fixed Offshore Platforms.}
\end{frame}

\begin{frame}{Justificación de Daños: Fenomenología Zonal (2/2)}
    \centering
    \textbf{Evidencia Forense:} Tasas de corrosión medidas en campo.
    
    \vspace{0.3cm}
    
    \begin{table}
        \centering
        \large 
        \caption{Tasas de Pérdida de Espesor por Zona (Datos Forenses HSE)}
        \begin{tabular}{lcc}
            \toprule
            \textbf{Zona Vertical} & \textbf{Tasa Real ($mm/a\tilde{n}o$)} & \textbf{Riesgo Estructural} \\
            \midrule
            \rowcolor{red!10} \textbf{Splash Zone} & \textbf{0.8 - 1.2} & \textbf{CRÍTICO (Máx $\Delta t$)} \\
            Marea (Tidal) & 0.4 - 0.6 & Alto \\
            Atmosférica & 0.1 - 0.3 & Medio \\
            Sumergida & $< 0.1$ & Bajo (Controlado por CP) \\
            \bottomrule
        \end{tabular}
    \end{table}
\end{frame}

% 2. ABOLLADURAS (Después)
\subsection{Abolladuras}
\begin{frame}{Justificación de Daños: Etiología y Severidad}
    \centering
    \small
    \textbf{Caracterización Forense:} El origen del impacto (etiología) define la morfología del daño y su riesgo asociado.
    
    \vspace{0.1cm}
    
    \begin{table}
        \centering
        \footnotesize % Fuente ajustada para evitar desbordamiento
        \renewcommand{\arraystretch}{1.2}
        \caption{Clasificación de Daños por Causa Raíz (Datos WOAD/HSE)}
        \begin{tabular}{l>{\raggedright\arraybackslash}p{3.8cm}l}
            \toprule
            \textbf{Causa Raíz} & \textbf{Morfología del Daño} & \textbf{Ubicación Típica} \\
            \midrule
            \rowcolor{red!10} \textbf{Colisión Buque} & \textbf{Abolladura Alargada + Arqueo Global (\textit{Bowing})} & \textbf{Splash Zone ($\pm 10$m)} \\
            Objetos Caídos & Abolladura Aguda ("Knife-edge") / Perforación & Miembros Horizontales \\
            Instalación & Abolladura Interna (\textit{Bulging}) & Piernas / Elementos diagonales \\
            \bottomrule
        \end{tabular}
    \end{table}

    \vfill % Empuja el contenido al fondo
    \hrule \vspace{0.1cm}
    \scriptsize
    \textbf{Fuentes de Datos:} \\
    \textbf{WOAD:} \textit{Worldwide Offshore Accident Databank} (Base de datos global de accidentes). \\
    \textbf{HSE:} \textit{Health and Safety Executive} (Agencia de seguridad del Reino Unido). \\
    \vspace{0.1cm}
    \textit{*Nota: El estudio se centra en el daño por impacto lateral (Colisión), que representa la mayor amenaza para la integridad global en la zona de salpicadura.}
\end{frame}

% Slide 9: Aportación Novedosa (ICD)
\section[ICD]{Índice de Calidad de Detección (ICD)}
\begin{frame}{Aportación Novedosa: Índice de Calidad de Detección (ICD)}
    \begin{block}{Definición ICD}
        Métrica híbrida optimizada evolutivamente que pondera:
        \begin{itemize}
            \item Sensibilidad de modos de vibración de orden superior.
            \item Robustez ante la incertidumbre.
        \end{itemize}
    \end{block}
    \vspace{0.5cm}
    \begin{alertblock}{Optimización}
        El Algoritmo Genético no solo busca el daño, sino que optimiza los pesos de ponderación del ICD para maximizar la detectabilidad.
    \end{alertblock}
\end{frame}

% Slide 9.5: Formulación Matemática del ICD
\begin{frame}{Formulación Matemática del ICD}
    \begin{block}{Ecuación General}
        El ICD se define como el producto de tres factores normalizados:
        \begin{equation}
            \text{ICD} = D \times C_{\text{norm}}(\delta) \times P_{\text{FP}}(N_{\text{FP}})
        \end{equation}
        \vspace{0.1cm}
        \centering \small \textbf{Interpretación:} $\text{ICD} \in [0, 1]$ \quad (Donde $1.0 =$ Detección perfecta).
    \end{block}

    \vspace{0.2cm}

    \begin{columns}[t]
        % Columna 1: Factor D
        \begin{column}{0.32\textwidth}
            \textbf{1. Éxito de Localización ($D$)}
            \begin{itemize} \footnotesize
                \item \textbf{1.0:} Detección exacta.
                \item \textbf{0.5:} Nodo adyacente (valor parcial).
                \item \textbf{0.0:} Fallo o ubicación errónea.
            \end{itemize}
        \end{column}

        % Columna 2: Factor C_norm
        \begin{column}{0.32\textwidth}
            \textbf{2. Confianza ($C_{\text{norm}}$)}
            \begin{itemize} \footnotesize
                \item Escalamiento logarítmico.
                \item Reconoce la dificultad de detectar daños incipientes ($\alpha=0.1$).
            \end{itemize}
             $$\frac{\ln(1 + \alpha \delta)}{\ln(1 + \alpha \delta_{\max})}$$
        \end{column}

        % Columna 3: Factor P_FP
        \begin{column}{0.32\textwidth}
            \textbf{3. Penalización ($P_{\text{FP}}$)}
            \begin{itemize} \footnotesize
                \item Decaimiento exponencial.
            \end{itemize}
            $$e^{- N_{\text{FP}}}$$
        \end{column}
    \end{columns}
\end{frame}

% Slide 10: Resultados - Abolladura (1/2)
\section[Resultados]{Resultados}
\subsection{Abolladura}
\begin{frame}{Resultados: Detección de Abolladura (1/2)}
    \centering
    \includegraphics[width=\textwidth, height=0.62\textheight, keepaspectratio]{005_abolladura_comparativa_global_de_ICD_vs_daño_por_zona_mudline_medio_y_splash_zone.png}
    \vspace{0.2cm}
    \captionof{figure}{Comparativa Global ICD vs Daño}
\end{frame}

% Slide 11: Resultados - Abolladura (2/2)
\begin{frame}{Resultados: Detección de Abolladura (2/2)}
    \centering
    \includegraphics[width=\textwidth, height=0.62\textheight, keepaspectratio]{006_abolladura_ICD_por_zona_desglozadp_por_tipo_de_elemento_desde_mudline_medio_y_splash_zone.png}
    \vspace{0.2cm}
    \captionof{figure}{Desglose por Zona y Tipo de Elemento}
\end{frame}

% Slide 12: Resultados - Corrosión (1/2)
\subsection{Corrosión}
\begin{frame}{Resultados: Detección de Corrosión (1/2)}
    \centering
    \includegraphics[width=\textwidth, height=0.62\textheight, keepaspectratio]{005_corrosion_comparativa_global_de_ICD_vs_daño_por_zona_mudline_medio_y_splash_zone.png}
    \vspace{0.2cm}
    \captionof{figure}{Comparativa Global ICD vs Corrosión}
\end{frame}

% Slide 13: Resultados - Corrosión (2/2)
\begin{frame}{Resultados: Detección de Corrosión (2/2)}
    \centering
    \includegraphics[width=\textwidth, height=0.62\textheight, keepaspectratio]{006_corrosion_ICD_por_zona_desglozadp_por_tipo_de_elemento_desde_mudline_medio_y_splash_zone.png}
    \vspace{0.2cm}
    \captionof{figure}{Desglose por Zona}
\end{frame}

% Slide 12: Estado de la Publicación
\section{Publicación y Conclusiones}
\begin{frame}{Estatus de Publicación JCR}
    \small
    \begin{itemize}
        \setlength\itemsep{0.3em} % Espacio entre puntos reducido

        \item \textbf{Título:} \\
        \textit{"Proposal of a Detection Quality Index (DQI) for Damage Identification in Jacket Platforms Using Genetic Algorithms."}

        \item \textbf{Revista Objetivo:} \\
        \textit{Journal of Civil Structural Health Monitoring} (Q1).

        \item \textbf{Estatus Actual:}
        \begin{itemize}
            \item Resultados del ICD consolidados.
            \item Artículo en proceso de redacción y formato.
            \item Requisito obligatorio para la graduación.
        \end{itemize}

        \item \textbf{Antecedentes y Perspectiva:} \\
        El manuscrito inicial fue rechazado en la revista \textit{Ocean Engineering}. Sin embargo, dicha versión carecía de la validación robusta actual. La integración del nuevo \textbf{Índice de Calidad de Detección (ICD)} subsana las limitaciones previas y fortalece la contribución científica, asegurando una propuesta sólida para este nuevo envío.
    \end{itemize}
\end{frame}

% Slide 13: Comentarios Finales
\subsection{Comentarios Finales}
\begin{frame}{Comentarios Finales y Siguientes Pasos}
    \begin{enumerate}
        \item \textbf{Validación:} El modelo simplificado de daño (abolladura/corrosión) demuestra ser computacionalmente eficiente y representativo.
        \item \textbf{Ruta Crítica (8º Semestre):}
        \begin{itemize}
            \item Envío y revisión del artículo JCR.
            \item Escritura final de la tesis.
            \item Defensa de grado.
            \item Profundizar en el estudio, generación y análisis de resultados para daños por grietas en la base (fagita) y deflexiones excesivas, aplicando la misma metodología.
        \end{itemize}
    \end{enumerate}
\end{frame}

\end{document}
