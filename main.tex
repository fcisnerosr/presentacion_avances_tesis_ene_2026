% --- CONFIGURACIÓN CRÍTICA PARA TABLAS ---
\PassOptionsToPackage{table}{xcolor}

\documentclass{beamer}

\usetheme{Madrid}
\useoutertheme{miniframes}
\usecolortheme{whale}
\setbeamerfont{section in head/foot}{size=\fontsize{6}{7}\selectfont}

\usepackage[utf8]{inputenc}
\usepackage[spanish,es-tabla]{babel}
\decimalpoint
\usepackage{graphicx}
\usepackage{booktabs}
\usepackage{amsmath}
\usepackage{array} % Para formateo avanzado de tablas

% Path configuration
\graphicspath{{./figs/}{./figs/resultados/abolladura/}{./figs/resultados/corrosion/}}

% Metadata
\title[Detección de Daño en Plataformas]{Desarrollo de una metodología para la detección de daño en plataformas marinas fijas por medio de análisis de vibraciones}
\author[M. I. Francisco Cisneros]{M. I. Francisco Cisneros \\ \small{Doctorante del 7mo Semestre}}
\institute[]{Posgrado en Ingeniería - UNAM}
\date{Viernes 23 de enero de 2026}

\begin{document}

% Estructura de secciones para barra de navegación
\section[Intro]{Introducción}

% Slide 1: Carátula Institucional
\begin{frame}[plain]
    \centering
    \includegraphics[height=1.3cm]{000_logo.png}
    \vspace{0.2cm}

    \vspace{0.1cm}
    {\footnotesize Coordinación Académica del Posgrado} \\
    {\footnotesize Dirección de Desarrollo de Talento}

    \vspace{0.6cm}

    {\bfseries \textit{EVALUACIÓN DE DESEMPEÑO}} \\
    \vspace{0.1cm}
    Trabajo de Investigación de Doctorado

    \vspace{0.4cm}

    \begin{beamercolorbox}[sep=8pt,center,shadow=true,rounded=true]{title}
        \usebeamerfont{title}Desarrollo de una metodología para la detección de daño en plataformas marinas fijas por medio de análisis de vibraciones\par
    \end{beamercolorbox}

    \vspace{0.4cm}

    \textbf{M. I. Francisco Cisneros}

    \vfill

    {\footnotesize
    Directores: Dr. Iván Félix González y Dr. Rolando Salgado Estrada \\
    \vspace{0.1cm}
    Periodo: Invierno 2025
    }
    \vspace{0.2cm}
\end{frame}

% Slide 2: Introducción
\subsection{Alcance y Problemática}
\begin{frame}{Introducción y Alcance}
    \begin{itemize}
        \item \textbf{Contexto:} Avances finales previos a la defensa de tesis (Evaluación de 7º Semestre).
        \vspace{0.5cm}
        \item \textbf{Problemática:}
        \begin{itemize}
            \item Crisis de mantenimiento en infraestructura envejecida.
            \item "Data Scarcity": Escasez de datos reales de daño que inviabiliza el uso puro de algoritmos de Deep Learning.
        \end{itemize}
        \vspace{0.5cm}
        \item \textbf{Solución Propuesta:}
        \begin{itemize}
            \item Hibridación de Algoritmos Genéticos (AG) con Modelos de Elemento Finito (FEM).
        \end{itemize}
    \end{itemize}
\end{frame}

% Slide 3: Metodología
\section[Metodología]{Metodología}
\subsection{Flujo General}
\begin{frame}{Metodología Propuesta (1/3): Flujo General}
    \begin{figure}
        \centering
        \includegraphics[width=\textwidth, height=0.8\textheight, keepaspectratio]{005_metogologia_propuesta_para_identificar_daños_en_plataformas_reales.jpg}
        \caption{Flujo de identificación en plataformas reales.}
    \end{figure}
\end{frame}

% Slide 4: Esquema Computacional
\subsection{Algoritmo Genético}
\begin{frame}{Metodología Propuesta (2/3): Esquema Computacional}
    \begin{figure}
        \centering
        \includegraphics[width=\textwidth, height=0.8\textheight, keepaspectratio]{006_metodología_del_AG_para_localizar_daños_a_nivel_computacional.jpg}
        \caption{Esquema computacional del AG.}
    \end{figure}
\end{frame}

% Slide 5: Mecánica AG
\begin{frame}{Metodología Propuesta (3/3): El Algoritmo Genético}
    \begin{figure}
        \centering
        \includegraphics[width=\textwidth, height=0.8\textheight, keepaspectratio]{015_diagrama_como_funciona_el_AG_para_la_sumas_funcion_P_y_como_se_relaciona_con_operadores_geneticos_del_AG.jpg}
        \caption{Mecánica del AG: función objetivo y operadores genéticos.}
    \end{figure}
\end{frame}

% --- SECCIÓN DE INDICADORES VIBRATORIOS (Recuperada) ---
\subsection{Indicadores Vibratorios}

% Slide DIs 1: Resumen
\begin{frame}{Fusión de Datos: Indicadores Vibratorios (DIs)}
    \footnotesize 
    \textbf{Estrategia de Fusión:}
    El AG no evalúa un solo parámetro, sino que optimiza un vector de pesos $\boldsymbol{\alpha}$ para minimizar el error mediante una \textbf{suma ponderada} de 8 indicadores distintos:
    
    \begin{equation}
        \text{Función Objetivo} \approx \min \sum_{j=1}^{8} \alpha_j \cdot DI_j
    \end{equation}
    
    \vspace{0.1cm}
    
    \begin{columns}[t]
        \begin{column}{0.48\textwidth}
            \begin{block}{Basados en Modos ($\phi$)}
                \begin{itemize} \scriptsize
                    \item \textbf{$DI_1$ (COMAC):} Correlación de vectores modales.
                    \item \textbf{$DI_2$ (Diff Abs):} Diferencia absoluta de formas modales.
                    \item \textbf{$DI_3$ (Razón Rel):} Relación entre vectores dañados e intactos.
                \end{itemize}
            \end{block}
        \end{column}
        
        \begin{column}{0.48\textwidth}
            \begin{block}{Basados en Flexibilidad ($F$)}
                \begin{itemize} \scriptsize
                    \item \textbf{$DI_4$ (Diff Flex):} Cambio en la diagonal de la matriz.
                    \item \textbf{$DI_5$ (Razón Flex):} Relación de flexibilidades.
                    \item \textbf{$DI_6$ (Var \%):} Porcentaje de variación.
                    \item \textbf{$DI_7$ (Z-Score):} Estandarización estadística.
                    \item \textbf{$DI_8$ (Probabilidad):} T. Gaussiana normalizada.
                \end{itemize}
            \end{block}
        \end{column}
    \end{columns}
    
    \vspace{0.2cm}
    \centering
    \scriptsize \textit{*El AG encuentra qué peso ($\alpha_j$) darle a cada $DI_j$ para maximizar la detección.}
\end{frame}

% Slide DIs 2: Fórmulas Modales
\begin{frame}{Formulación Matemática: Indicadores Modales ($DI_1 - DI_3$)}
    \small
    \textbf{1. COMAC ($DI_1$):}
    Evalúa la correlación punto a punto de los modos. (1 = Correlación perfecta, 0 = Sin correlación).
    \begin{equation}
        \text{DI}_1(j) = 1 - \sqrt{ \frac{ \left( \sum_{i=1}^{N_m} |\phi_{u,ij}| |\phi_{d,ij}| \right)^2 }{ \sum_{i=1}^{N_m} \phi_{u,ij}^2 \sum_{i=1}^{N_m} \phi_{d,ij}^2 } }
    \end{equation}

    \vspace{0.2cm}

    \textbf{2. Diferencia Absoluta ($DI_2$) y Razón Relativa ($DI_3$):}
    Cuantifican el desplazamiento de los vectores modales $\boldsymbol{\phi}$.
    \begin{align}
        \text{DI}_2(j) &= \mathcal{R}_j \left( \sum_{i=1}^{N_m} | \boldsymbol{\phi}_{u,i} - \boldsymbol{\phi}_{d,i} | \right) \\
        \text{DI}_3(j) &= \mathcal{R}_j \left( \sum_{i=1}^{N_m} \left| \frac{\boldsymbol{\phi}_{d,i}}{\boldsymbol{\phi}_{u,i} + \epsilon} - 1 \right| \right)
    \end{align}
    
    \footnotesize \textit{*Donde $u$ es estado intacto, $d$ dañado y $\epsilon$ es un factor de regularización.}
\end{frame}

% Slide DIs 3: Fórmulas Flexibilidad
\begin{frame}{Formulación Matemática: Flexibilidad y Estadística ($DI_4 - DI_8$)}
    \small
    \textbf{Basados en la Matriz de Flexibilidad ($\mathbf{F} \approx \boldsymbol{\Phi} \boldsymbol{\Omega}^{-2} \boldsymbol{\Phi}^T$):}
    \begin{itemize}
        \item \textbf{Diferencia ($DI_4$):} $| \text{diag}(\mathbf{F}_d) - \text{diag}(\mathbf{F}_u) |$
        \item \textbf{Razón ($DI_5$):} $\left| \frac{\text{diag}(\mathbf{F}_d)}{\text{diag}(\mathbf{F}_u) + \epsilon} - 1 \right|$
        \item \textbf{Variación \% ($DI_6$):} Versión porcentual de la razón ($DI_5 \times 100$).
    \end{itemize}

    \vspace{0.2cm}
    \hrule
    \vspace{0.2cm}

    \textbf{Estandarización Estadística ($DI_7$ y $DI_8$):}
    Normalizan el daño asumiendo una distribución Gaussiana para resaltar anomalías (outliers).
    \begin{equation}
        \text{DI}_7(j) = \frac{\Delta F_j - \mu_{\Delta F}}{\sigma_{\Delta F}} \quad \rightarrow \quad \text{DI}_8(j) = 1 - 2\left( 1 - \Phi_{\text{cdf}}(|\text{DI}_7|) \right)
    \end{equation}
\end{frame}

% --- EFECTOS INERCIALES Y AMBIENTALES (Movido Antes de Justificación) ---
\section[Efectos]{Efectos Inerciales y Ambientales}
\begin{frame}{Efectos Inerciales y Ambientales (1/2)}
    \textbf{Consideraciones:} Inclusión de masa añadida hidrodinámica y crecimiento marino (biofouling).
    \begin{figure}
        \centering
        \includegraphics[width=0.85\textwidth, height=0.52\textheight, keepaspectratio]{001_modelo_plataforma_con_elemento_tubular_masa_adherida_y_masa_atrapada.png}
        \caption{Detalle de Masa Hidrodinámica (Adherida y Atrapada).}
    \end{figure}
\end{frame}

\begin{frame}{Efectos Inerciales y Ambientales (2/2)}
    \begin{figure}
        \centering
        \includegraphics[width=\textwidth, height=0.65\textheight, keepaspectratio]{001_modelo_plataforma_con_elemento_tubular_masa_de_crecimiento_marino.png}
        \caption{Modelado del Crecimiento Marino (Biofouling) y coeficientes rugosos.}
    \end{figure}

    \vfill 
    \hrule \vspace{0.1cm}
    \tiny
    \textbf{Fuente:} American Petroleum Institute. (2014). \textit{Planning, designing, and constructing fixed offshore platforms - working stress design}. Washington, Estados Unidos.
\end{frame}

\begin{frame}{Modelado de Daño: Corrosión Longitudinal}
    \small
    \textbf{Caracterización Geométrica:}
    Se modela como una reducción uniforme del espesor de pared ($t$) aplicada en tramos discretos del elemento, simulando la pérdida de material en zonas críticas (ej. \textit{Splash Zone}).

    \begin{figure}
        \centering
        \includegraphics[width=\textwidth, height=0.6\textheight, keepaspectratio]{004_modelo_elemento_tubular_con_caracterizacion_corrosion_longitudinal_donde_muestran_tramos_con_reduccion_espesor.png}
        \caption{Perfil de reducción de espesor ($t_{loss}$) a lo largo del elemento tubular.}
    \end{figure}
\end{frame}

\begin{frame}{Modelado de Daño: Abolladura}
    \textbf{Caracterización Geométrica:}
    Se modela como una imperfección geométrica local definida por la profundidad ($d/D$) y la longitud del daño, alterando la inercia local del elemento.

    \begin{figure}
        \centering
        % ATENCIÓN: Reemplaza 'nombre_de_tu_imagen_abolladura.png' con tu archivo real
        \includegraphics[width=\textwidth, height=0.55\textheight, keepaspectratio]{004_seccion_transversal_de_elemento_tubular_parametros_abolladura_con_elementos_discretizados.png}
        \caption{Representación esquemática del elemento tubular con daño por abolladura.}
    \end{figure}
\end{frame}

\begin{frame}{Modelado de Daño: Interpolación Longitudinal (Abolladura)}
    \small
    \textbf{Continuidad Numérica:}
    Para evitar discontinuidades abruptas en la matriz de rigidez global, la variación del momento de inercia ($I$) a lo largo de la zona dañada se suaviza matemáticamente.

    \begin{figure}
        \centering
        % Ajustamos la imagen al ancho disponible manteniendo la proporción
        \includegraphics[width=\textwidth, height=0.6\textheight, keepaspectratio]{004_modelo_elemento_tubular_seccion_longitudinal_parametros_abolladura_con_funciones_de_interpolacion.png}
        \caption{Interpolación de la reducción de inercia mediante polinomios de 4º grado.}
    \end{figure}
\end{frame}

% --- JUSTIFICACIÓN Y CARACTERIZACIÓN (Ordenado: Corrosión -> Abolladura) ---
\section[Justificación]{Justificación y Caracterización}

% 1. CORROSIÓN
\subsection{Corrosión}
\begin{frame}{Justificación de Daños: Fenomenología Zonal (1/2)}
    \small
    \textbf{¿Por qué modelar Corrosión Uniforme?}
    Aunque la corrosión puede ser local, para efectos de rigidez global ($EI$) en estructuras Jacket, la normativa (API/ISO) valida el modelo de \textbf{reducción uniforme de espesor} ($t_{loss}$) como el mecanismo que gobierna la pérdida de capacidad de carga.

    \vspace{0.3cm}
    \textbf{Distribución Vertical de la Severidad:}
    La pérdida de espesor \textbf{no es constante} en toda la altura. Se concentra drásticamente en la zona de humectación y secado (Splash Zone).

    \vfill 
    \hrule \vspace{0.1cm}
    \tiny \color{gray}
    \textbf{Fuentes Normativas:} \\
    \textbullet\ \textbf{ISO 19902:2007:} \textit{Petroleum and natural gas industries -- Fixed steel offshore structures.} (Sec. 16.3 Structural Integrity). \\
    \textbullet\ \textbf{API RP 2A-WSD:} \textit{Recommended Practice for Planning, Designing and Constructing Fixed Offshore Platforms.}
\end{frame}

\begin{frame}{Justificación de Daños: Fenomenología Zonal (2/2)}
    \centering
    \textbf{Evidencia Forense:} Tasas de corrosión medidas en campo.
    
    \vspace{0.3cm}
    
    \begin{table}
        \centering
        \large 
        \caption{Tasas de Pérdida de Espesor por Zona (Datos Forenses HSE)}
        \begin{tabular}{lcc}
            \toprule
            \textbf{Zona Vertical} & \textbf{Tasa Real ($mm/a\tilde{n}o$)} & \textbf{Riesgo Estructural} \\
            \midrule
            \rowcolor{red!10} \textbf{Splash Zone} & \textbf{0.8 - 1.2} & \textbf{CRÍTICO (Máx $\Delta t$)} \\
            Marea (Tidal) & 0.4 - 0.6 & Alto \\
            Atmosférica & 0.1 - 0.3 & Medio \\
            Sumergida & $< 0.1$ & Bajo (Controlado por CP) \\
            \bottomrule
        \end{tabular}
    \end{table}
\end{frame}

\subsection{Abolladuras}

\begin{frame}{Justificación de Daños: Etiología y Severidad}
    \centering
    \small
    \textbf{Caracterización Forense:} El origen del impacto (etiología) define la morfología del daño.
    
    \vspace{0.1cm}
    
    \begin{table}
        \centering
        \footnotesize 
        \renewcommand{\arraystretch}{1.2}
        \caption{Clasificación de Daños por Causa Raíz (Datos WOAD/HSE)}
        \begin{tabular}{l>{\raggedright\arraybackslash}p{3.8cm}l}
            \toprule
            \textbf{Causa Raíz} & \textbf{Morfología del Daño} & \textbf{Ubicación Típica} \\
            \midrule
            \rowcolor{red!10} \textbf{Colisión Buque} & \textbf{Abolladura Alargada + Arqueo Global (\textit{Bowing})} & \textbf{Splash Zone ($\pm 10$m)} \\
            Objetos Caídos & Abolladura Aguda & Miembros Horizontales \\
            Instalación & Abolladura Interna & Piernas / Elementos diagonales \\
            \bottomrule
        \end{tabular}
    \end{table}

    \vfill
    \hrule \vspace{0.1cm}
    \scriptsize
    \textbf{Fuentes de Datos:} \textbf{WOAD:} \textit{Worldwide Offshore Accident Databank}; \textbf{HSE:} \textit{Health and Safety Executive}.
\end{frame}

% --- ICD Y RESULTADOS ---
\section[ICD]{Índice de Calidad de Detección (ICD)}
\begin{frame}{Formulación Matemática del ICD}
    \begin{block}{Ecuación General}
        El ICD se define como el producto de tres factores normalizados:
        \begin{equation}
            \text{ICD} = D \times C_{\text{norm}}(\delta) \times P_{\text{FP}}(N_{\text{FP}})
        \end{equation}
        \vspace{0.1cm}
        \centering \small \textbf{Interpretación:} $\text{ICD} \in [0, 1]$ \quad (Donde $1.0 =$ Detección perfecta).
    \end{block}

    \vspace{0.2cm}

    \begin{columns}[t]
        \begin{column}{0.32\textwidth}
            \textbf{1. Éxito ($D$)}
            \begin{itemize} \footnotesize
                \item \textbf{1.0:} Exacta.
                \item \textbf{0.5:} Adyacente.
                \item \textbf{0.0:} Fallo.
            \end{itemize}
        \end{column}

        \begin{column}{0.34\textwidth}
            \textbf{2. Confianza ($C_{\text{norm}}$)}
            \begin{itemize} \scriptsize
                \item Escalamiento logarítmico ($\alpha=0.1$).
            \end{itemize}
            \vspace{0.1cm}
            \centering \small
            $\displaystyle \frac{\ln(1 + \alpha \delta)}{\ln(1 + \alpha \delta_{\max})}$
        \end{column}

        \begin{column}{0.30\textwidth}
            \textbf{3. Penalización ($P_{\text{FP}}$)}
            \begin{itemize} \scriptsize
                \item Decaimiento exp. ($\beta=0.15$).
            \end{itemize}
            \vspace{0.1cm}
            \centering \small
            $\displaystyle e^{- \beta N_{\text{FP}}}$
        \end{column}
    \end{columns}
\end{frame}

% Caso de Estudio
\section[Caso]{Caso de Estudio}
\begin{frame}{Caso de Estudio: Plataforma Tipo Jacket}
    \begin{itemize}
        \item Plataforma marina fija discretizada mediante elementos viga.
    \end{itemize}
    \vspace{0.3cm}
    \begin{columns}
        \column{0.5\textwidth}
        \begin{figure}
            \centering
            \includegraphics[width=\linewidth, height=0.7\textheight, keepaspectratio]{007_modelo_vista_fronta_de_plataforma_tipo_jacket_de_caso_de_estudio.png}
            \caption{Vista Frontal}
        \end{figure}
        
        \column{0.5\textwidth}
        \begin{figure}
            \centering
            \includegraphics[width=\linewidth, height=0.7\textheight, keepaspectratio]{007_modeo_vista_3d_de_plataforma_tipo_jacket_de_caso_de_estudio.png}
            \caption{Vista 3D}
        \end{figure}
    \end{columns}
\end{frame}



\section[Resultados]{Resultados}
\subsection{Abolladura}
\begin{frame}{Resultados: Detección de Abolladura (1/2)}
    \begin{figure}
        \centering
        \includegraphics[width=\textwidth, height=0.77\textheight, keepaspectratio]{005_abolladura_comparativa_global_de_ICD_vs_daño_por_zona_mudline_medio_y_splash_zone.png}
        \caption{Comparativa Global ICD vs Daño}
    \end{figure}
\end{frame}

\begin{frame}{Resultados: Detección de Abolladura (2/2)}
    \begin{figure}
        \centering
        \includegraphics[width=\textwidth, height=0.77\textheight, keepaspectratio]{006_abolladura_ICD_por_zona_desglozadp_por_tipo_de_elemento_desde_mudline_medio_y_splash_zone.png}
        \caption{Desglose por Zona y Tipo de Elemento}
    \end{figure}
\end{frame}

\subsection{Corrosión}
\begin{frame}{Resultados: Detección de Corrosión (1/2)}
    \begin{figure}
        \centering
        \includegraphics[width=\textwidth, height=0.77\textheight, keepaspectratio]{005_corrosion_comparativa_global_de_ICD_vs_daño_por_zona_mudline_medio_y_splash_zone.png}
        \caption{Comparativa Global ICD vs Corrosión}
    \end{figure}
\end{frame}

\begin{frame}{Resultados: Detección de Corrosión (2/2)}
    \begin{figure}
        \centering
        \includegraphics[width=\textwidth, height=0.78\textheight, keepaspectratio]{006_corrosion_ICD_por_zona_desglozadp_por_tipo_de_elemento_desde_mudline_medio_y_splash_zone.png}
        \caption{Desglose por Zona}
    \end{figure}
\end{frame}

\section[Cierre]{Publicación y Conclusiones}
\begin{frame}{Estatus de Publicación JCR (Journal Citation Reports)}
    \small 
    \begin{itemize}
        \setlength\itemsep{0.4em} 

        \item \textbf{Título Tentativo:} \\
        \textit{"Proposal of a Detection Quality Index (DQI) for Damage Identification in Jacket Platforms Using Genetic Algorithms."}

        \item \textbf{Revista Objetivo:} \\
        \textit{Journal of Civil Structural Health Monitoring} (Q1 - JCR).

        \item \textbf{Estatus Actual:}
        \begin{itemize} \footnotesize
            \item Resultados del ICD consolidados.
            \item Artículo en proceso de redacción y formato.
            \item Requisito obligatorio para la graduación.
        \end{itemize}

        \item \textbf{Antecedentes y Perspectiva:} \\
        \footnotesize 
        El manuscrito inicial fue rechazado en la revista \textit{Ocean Engineering}. Sin embargo, dicha versión carecía de la validación robusta actual. La integración del nuevo \textbf{Índice de Calidad de Detección (ICD)} subsana las limitaciones previas y fortalece la contribución científica, asegurando una propuesta sólida para este nuevo envío.
    \end{itemize}
\end{frame}

\subsection{Comentarios Finales}
\begin{frame}{Comentarios Finales y Siguientes Pasos}
    \begin{enumerate}
        \item \textbf{Validación:} El modelo simplificado de daño (abolladura/corrosión) demuestra ser computacionalmente eficiente y representativo.
        \item \textbf{Ruta Crítica (8º Semestre):}
        \begin{itemize}
            \item Envío y revisión del artículo JCR.
            \item Escritura final de la tesis.
            \item Defensa de grado.
            \item Profundizar en el estudio, generación y análisis de resultados para daños por grietas en la base (fatiga) y deflexiones excesivas, aplicando la misma metodología.
        \end{itemize}
    \end{enumerate}
\end{frame}

\end{document}
