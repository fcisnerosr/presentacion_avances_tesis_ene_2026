\documentclass{beamer}

\usetheme{Madrid}
\usecolortheme{whale}

\usepackage[utf8]{inputenc}
\usepackage[spanish]{babel}
\usepackage{graphicx}
\usepackage{subcaption}
\usepackage{booktabs}
\usepackage{amsmath}

% Path configuration
\graphicspath{{./figs/}{./figs/resultados/abolladura/}{./figs/resultados/corrosion/}}

% Metadata
\title[Detección de Daño en Plataformas]{Desarrollo de una metodología para la detección de daño en plataformas marinas fijas por medio de análisis de vibraciones}
\author[M. I. Francisco Cisneros]{M. I. Francisco Cisneros \\ \small{Doctorante del 7mo Semestre}}
\institute[]{Posgrado en Ingeniería - UNAM}
\date{Viernes 23 de enero de 2026}

\begin{document}

% Slide 1
\begin{frame}
    \titlepage
\end{frame}

% Slide 2: Introducción y Alcance
\begin{frame}{Introducción y Alcance}
    \begin{itemize}
        \item \textbf{Contexto:} Avances finales previos a la defensa de tesis (Evaluación de 7º Semestre).
        \vspace{0.5cm}
        \item \textbf{Problemática:}
        \begin{itemize}
            \item Crisis de mantenimiento en infraestructura envejecida.
            \item "Data Scarcity": Escasez de datos reales de daño que inviabiliza el uso puro de algoritmos de Deep Learning.
        \end{itemize}
        \vspace{0.5cm}
        \item \textbf{Solución Propuesta:}
        \begin{itemize}
            \item Sistema SHM robusto ante la falta de sensores.
            \item Hibridación de Algoritmos Genéticos (AG) con Modelos de Elemento Finito (FEM).
        \end{itemize}
    \end{itemize}
\end{frame}

% Slide 3: Metodología Propuesta (1/3)
\begin{frame}{Metodología Propuesta (1/3): Flujo General}
    \centering
    \includegraphics[width=\textwidth, height=0.8\textheight, keepaspectratio]{005_metogologia_propuesta_para_identificar_daños_en_plataformas_reales.jpg}
    \vspace{0.2cm}
    \captionof{figure}{Flujo de identificación en plataformas reales.}
\end{frame}

% Slide 4: Metodología Propuesta (2/3)
\begin{frame}{Metodología Propuesta (2/3): Esquema Computacional}
    \centering
    \includegraphics[width=\textwidth, height=0.8\textheight, keepaspectratio]{006_metodología_del_AG_para_localizar_daños_a_nivel_computacional.jpg}
    \vspace{0.2cm}
    \captionof{figure}{Esquema computacional del AG.}
\end{frame}

% Slide 5: Metodología Propuesta (3/3)
\begin{frame}{Metodología Propuesta (3/3): El Algoritmo Genético}
    \centering
    \includegraphics[width=\textwidth, height=0.8\textheight, keepaspectratio]{015_diagrama_como_funciona_el_AG_para_la_sumas_funcion_P_y_como_se_relaciona_con_operadores_geneticos_del_AG.jpg}
    \captionof{figure}{Mecánica del AG: función objetivo y operadores genéticos.}
\end{frame}

% Slide 5: Caso de Estudio (Modelo Jacket)
\begin{frame}{Caso de Estudio: Plataforma Tipo Jacket}
    \begin{itemize}
        \item Plataforma marina fija discretizada mediante elementos viga.
    \end{itemize}
    \vspace{0.3cm}
    \begin{columns}
        \column{0.5\textwidth}
        \centering
        \includegraphics[width=\textwidth, height=0.6\textheight, keepaspectratio]{007_modelo_vista_fronta_de_plataforma_tipo_jacket_de_caso_de_estudio.png}
        \tiny{Vista Frontal}
        
        \column{0.5\textwidth}
        \centering
        \includegraphics[width=\textwidth, height=0.6\textheight, keepaspectratio]{007_modeo_vista_3d_de_plataforma_tipo_jacket_de_caso_de_estudio.png}
        \tiny{Vista 3D}
    \end{columns}
\end{frame}

% Slide 6: Modelado de Daño - Abolladura
\begin{frame}{Modelado de Daño: Abolladura}
    \textbf{Enfoque:} Reducción de rigidez localizada en elementos viga.
    \vspace{0.3cm}
    
    \begin{figure}
        \centering
        \begin{subfigure}[b]{0.5\textwidth}
            \centering
            \includegraphics[width=\textwidth]{004_seccion_transversal_de_elemento_tubular_parametros_abolladura_con_elementos_discretizados.png}
            \caption{\tiny Sección Transversal}
        \end{subfigure}
        \\[0.2cm]
        \begin{subfigure}[b]{0.95\textwidth}
            \centering
            \includegraphics[width=\textwidth]{004_modelo_elemento_tubula_seccion_longitudinal_parametros_abolladura.png}
            \caption{\tiny Parametrización}
        \end{subfigure}
    \end{figure}
\end{frame}

% Slide 7: Modelado de Daño - Corrosión
\begin{frame}{Modelado de Daño: Corrosión}
    \textbf{Enfoque:} Corrosión Uniforme por zonas (Splash Zone vs. Sumergida).
    \vspace{0.3cm}
    
    \begin{columns}
        \column{0.5\textwidth}
        \centering
        \includegraphics[width=0.9\textwidth, height=0.5\textheight, keepaspectratio]{002_elemento_tubular_con_corrosion_excesiva.png}
        \small{Ejemplo de corrosión severa.}
        
        \column{0.5\textwidth}
        \centering
        \includegraphics[width=0.9\textwidth, height=0.5\textheight, keepaspectratio]{004_caracterizacion_de_como_se_reduce_el_espesor_elemento_tubular_por_corrosion.png}
        \small{Reducción efectiva del espesor.}
    \end{columns}
\end{frame}

% Slide 8: Modificación de la Matriz de Masas
\begin{frame}{Efectos Inerciales y Ambientales}
    \textbf{Consideraciones:} Inclusión de masa añadida hidrodinámica y crecimiento marino (biofouling).
    \vspace{0.3cm}
    
    \begin{columns}
        \column{0.5\textwidth}
        \centering
        \includegraphics[width=0.8\textwidth]{001_modelo_plataforma_con_elemento_tubular_masa_adherida_y_masa_atrapada.png}
        \small{Masa Adherida/Atrapada}
        
        \column{0.5\textwidth}
        \centering
        \includegraphics[width=0.8\textwidth]{001_modelo_plataforma_con_elemento_tubular_masa_de_crecimiento_marino.png}
        \small{Crecimiento Marino}
    \end{columns}
\end{frame}

% Slide 9: Aportación Novedosa (ICD)
\begin{frame}{Aportación Novedosa: Índice de Calidad de Detección (ICD)}
    \begin{block}{Definición ICD}
        Métrica híbrida optimizada evolutivamente que pondera:
        \begin{itemize}
            \item Sensibilidad de modos de vibración de orden superior.
            \item Estabilidad numérica de la matriz de flexibilidad.
        \end{itemize}
    \end{block}
    \vspace{0.5cm}
    \begin{alertblock}{Optimización}
        El Algoritmo Genético no solo busca el daño, sino que optimiza los pesos de ponderación del ICD para maximizar la detectabilidad en escenarios de ruido.
    \end{alertblock}
\end{frame}

% Slide 10: Resultados - Abolladura (1/2)
\begin{frame}{Resultados: Detección de Abolladura (1/2)}
    \centering
    \includegraphics[width=\textwidth, height=0.75\textheight, keepaspectratio]{005_abolladura_comparativa_global_de_ICD_vs_daño_por_zona_mudline_medio_y_splash_zone.png}
    \vspace{0.2cm}
    \captionof{figure}{Comparativa Global ICD vs Daño}
\end{frame}

% Slide 11: Resultados - Abolladura (2/2)
\begin{frame}{Resultados: Detección de Abolladura (2/2)}
    \centering
    \includegraphics[width=\textwidth, height=0.75\textheight, keepaspectratio]{006_abolladura_ICD_por_zona_desglozadp_por_tipo_de_elemento_desde_mudline_medio_y_splash_zone.png}
    \vspace{0.2cm}
    \captionof{figure}{Desglose por Zona y Tipo de Elemento}
\end{frame}

% Slide 12: Resultados - Corrosión (1/2)
\begin{frame}{Resultados: Detección de Corrosión (1/2)}
    \centering
    \includegraphics[width=\textwidth, height=0.75\textheight, keepaspectratio]{005_corrosion_comparativa_global_de_ICD_vs_daño_por_zona_mudline_medio_y_splash_zone.png}
    \vspace{0.2cm}
    \captionof{figure}{Comparativa Global ICD vs Corrosión}
\end{frame}

% Slide 13: Resultados - Corrosión (2/2)
\begin{frame}{Resultados: Detección de Corrosión (2/2)}
    \centering
    \includegraphics[width=\textwidth, height=0.75\textheight, keepaspectratio]{006_corrosion_ICD_por_zona_desglozadp_por_tipo_de_elemento_desde_mudline_medio_y_splash_zone.png}
    \vspace{0.2cm}
    \captionof{figure}{Desglose por Zona}
\end{frame}

% Slide 12: Estado de la Publicación
\begin{frame}{Estatus de Publicación JCR}
    \begin{itemize}
        \item \textbf{Título Tentativo:} Metodología basada en ICD y Algoritmos Genéticos para detección de daño estructural.
        \item \textbf{Revista Objetivo:} Ingeniería Investigación y Tecnología.
        \item \textbf{Estatus:}
        \begin{itemize}
            \item Resultados de ICD consolidados.
            \item Artículo en proceso de redacción y formateo.
            \item Requisito indispensable para la graduación.
        \end{itemize}
    \end{itemize}
\end{frame}

% Slide 13: Comentarios Finales
\begin{frame}{Comentarios Finales y Siguientes Pasos}
    \begin{enumerate}
        \item \textbf{Validación:} El modelo simplificado de daño (abolladura/corrosión) demuestra ser computacionalmente eficiente y representativo.
        \item \textbf{Eficacia del ICD:} La métrica híbrida supera a los indicadores tradicionales en escenarios de ruido.
        \item \textbf{Ruta Crítica (8º Semestre):}
        \begin{itemize}
            \item Envío y revisión del artículo JCR.
            \item Escritura final de la tesis.
            \item Defensa de grado.
        \end{itemize}
    \end{enumerate}
\end{frame}

\end{document}